\documentclass[12pt]{article}
\usepackage{ragged2e} % load the package for justification
\usepackage{hyperref}
\usepackage[utf8]{inputenc}
\usepackage{pgfplots}
\usepackage{tikz}
\usetikzlibrary{fadings}
\usepackage{filecontents}
\usepackage{multirow}
\usepackage{amsmath}
\pgfplotsset{width=10cm,compat=1.17}
\setlength{\parskip}{0.75em} % Set the space between paragraphs
\usepackage{setspace}
\setstretch{1.2} % Adjust the value as per your preference
\usepackage[margin=2cm]{geometry} % Adjust the margin
\setlength{\parindent}{0pt} % Adjust the value for starting paragraph
\usetikzlibrary{arrows.meta}
\usepackage{mdframed}
\usepackage{float}

\usepackage{hyperref}

% to remove the hyperline rectangle
\hypersetup{
	colorlinks=true,
	linkcolor=black,
	urlcolor=blue
}

\usepackage{xcolor}
\usepackage{titlesec}
\usepackage{titletoc}
\usepackage{listings}
\usepackage{tcolorbox}
\usepackage{lipsum} % Example text package
\usepackage{fancyhdr} % Package for customizing headers and footers

\usepackage{algorithm}
\usepackage{algpseudocode}

% Define the orange color
\definecolor{myorange}{RGB}{255,65,0}
% Define a new color for "cherry" (dark red)
\definecolor{cherry}{RGB}{148,0,25}
\definecolor{codegreen}{rgb}{0,0.6,0}



%%%%%%%%%%%%%%%%%%%%%%%%%%%%%%%%%%%%%%%%%%%%%%%%%%%%%%%%%%%%%%%%%%%%%
% Apply the custom footer to all pages
\pagestyle{fancy}

% Redefine the header format
\fancyhead{}
\fancyhead[R]{\textcolor{orange!80!black}{\itshape\leftmark}}

\fancyhead[L]{\textcolor{black}{\thepage}}


% Redefine the footer format with a line before each footnote
\fancyfoot{}
\fancyfoot[C]{\footnotesize \textbf{Fady Abousifein}, McMaster University, MECHTRON 2MP3 - Programming for Mechatronics. \footnoterule}

% Redefine the footnote rule
\renewcommand{\footnoterule}{\vspace*{-3pt}\noindent\rule{0.0\columnwidth}{0.4pt}\vspace*{2.6pt}}

% Set the header rule color to orange
\renewcommand{\headrule}{\color{orange!80!black}\hrule width\headwidth height\headrulewidth \vskip-\headrulewidth}

% Set the footer rule color to orange (optional)
\renewcommand{\footrule}{\color{black}\hrule width\headwidth height\headrulewidth \vskip-\headrulewidth}

%%%%%%%%%%%%%%%%%%%%%%%%%%%%%%%%%%%%%%%%%%%%%%%%%%%%%%%%%%%%%%%%%%%%%


% Set the color for the section headings
\titleformat{\section}
{\normalfont\Large\bfseries\color{orange!80!black}}{\thesection}{1em}{}

% Set the color for the subsection headings
\titleformat{\subsection}
{\normalfont\large\bfseries\color{orange!80!black}}{\thesubsection}{1em}{}

% Set the color for the subsubsection headings
\titleformat{\subsubsection}
{\normalfont\normalsize\bfseries\color{orange!80!black}}{\thesubsubsection}{1em}{}


%%%%%%%%%%%%%%%%%%%%%%%%%%%%%%%%%%%%%%%%%%%%%%%%%%%%%%%%%%%%%%%%%%%%%
% Set the color for the table of contents
\titlecontents{section}
[1.5em]{\color{orange!80!black}}
{\contentslabel{1.5em}}
{}{\titlerule*[0.5pc]{.}\contentspage}

% Set the color for the subsections in the table of contents
\titlecontents{subsection}
[3.8em]{\color{orange!80!black}}
{\contentslabel{2.3em}}
{}{\titlerule*[0.5pc]{.}\contentspage}

% Set the color for the subsubsections in the table of contents
\titlecontents{subsubsection}
[6em]{\color{orange!80!black}}
{\contentslabel{3em}}
{}{\titlerule*[0.5pc]{.}\contentspage}


%%%%%%%%%%%%%%%%%%%%%%%%%%%%%%%%%%%%%%%%%%%%%%%%%%%%%%%%%%%%%%%%%%%%%
% set a format for the codes inside a box with C format
\lstset{
	language=C,
	basicstyle=\ttfamily,
	backgroundcolor=\color{blue!5},
	keywordstyle=\color{blue},
	commentstyle=\color{codegreen},
	stringstyle=\color{red},
	showstringspaces=false,
	breaklines=true,
	frame=single,
	rulecolor=\color{lightgray!35}, % Set the color of the frame
	numbers=none,
	numberstyle=\tiny,
	numbersep=5pt,
	tabsize=1,
	morekeywords={include},
	alsoletter={\#},
	otherkeywords={\#}
}




%\input listings.tex



% Define a command for inline code snippets with a colored and rounded box
\newtcbox{\codebox}[1][gray]{on line, boxrule=0.2pt, colback=blue!5, colframe=#1, fontupper=\color{cherry}\ttfamily, arc=2pt, boxsep=0pt, left=2pt, right=2pt, top=3pt, bottom=2pt}




\tikzset{%
	every neuron/.style={
		circle,
		draw,
		minimum size=1cm
	},
	neuron missing/.style={
		draw=none, 
		scale=4,
		text height=0.333cm,
		execute at begin node=\color{black}$\vdots$
	},
}



%%%%%%%%%%%%%%%%%%%%%%%%%%%%%%%%%%%%%%%%%%%%%%%%%%%%%%%%%%%%%%%%%%%%%

% Define a new tcolorbox style with default options
\tcbset{
	myboxstyle/.style={
		colback=orange!10,
		colframe=orange!80!black,
	}
}

% Define a new tcolorbox style with default options to print the output with terminal style


\tcbset{
	myboxstyleTerminal/.style={
		colback=blue!5,
		frame empty, % Set frame to empty to remove the fram
	}
}

\mdfdefinestyle{myboxstyleTerminal1}{
	backgroundcolor=blue!5,
	hidealllines=true, % Remove all lines (frame)
	leftline=false,     % Add a left line
}


\begin{document}
	
	\justifying
	
	\begin{center}
		\textbf{{\large Assignment 4}}
		
		\textbf{Developing Particle Swarm Optimization (PSO) in C with Case Study} 
		
		\textbf{Fady Abousifein}
		
		\textbf{400506836}
	\end{center}
	

	
	
	
	\section{Introduction}
	In this assignment we are tasked with implementing \textbf{Particle Swarm Optimization (PSO)}, which is a method of optimization modeled after the social behavior of birds flocking or fish schooling. It is often used when solving optimization problems at which normal mathematical methods may eithe not work or be too computationally taxing. 

PSO simulates the behavior of a large amount of particles each of which represent a possible solution to the optimization problem. These particles iterativley change their position based on their best positions and the global best position of all of the particles, eventually the swarm of particles will converge towards one position and thus the optimal solution.

	
	\section{Particle Swarm Optimization (PSO) Formulation}
    Before we start talking about how the algorithim works, we must first define the constants and variables that we will be using:
	\begin{itemize}
    \item $N$: Number of particles in the swarm.
    \item $D$: Dimensionality of the search space (number of variables).
    \item $\mathbf{x}_i = \begin{bmatrix} x_{i1}, x_{i2}, \dots, x_{iD} \end{bmatrix}^\top$: Position vector of the $i$th particle.
    \item $\mathbf{v}_i = \begin{bmatrix} v_{i1}, v_{i2}, \dots, v_{iD} \end{bmatrix}^\top$: Velocity vector of the $i$th particle.
    \item $p_i$: Personal best position of the $i$th particle.
    \item $g$: Global best position among all particles.
    \item $w$: Inertia weight, controlling the influence of the previous velocity.
    \item $c_1, c_2$: Cognitive coefficient, measuring the particle's tendency to return to its personal best position, and the Social coefficient, measuring the particle's tendency to follow the global best position. 
    \item $r_1, r_2$: Random numbers uniformly distributed in $[0, 1]$.
\end{itemize}
These are all the variables and constants that are significant to the PSO algorithm. The PSO algorithm is dependent on how the particles update their attributes, and these the the formulas that define how the particles update their attributes: 

\subsection*{Velocity Update}
The velocity of each particle is updated at each iteration as follows:
\[
\mathbf{v}_i^{(t+1)} = w \mathbf{v}_i^{(t)} + c_1 r_1 \left( \mathbf{p}_i - \mathbf{x}_i^{(t)} \right) + c_2 r_2 \left( \mathbf{g} - \mathbf{x}_i^{(t)} \right)
\]

\subsection*{Position Update}
The position of each particle is updated as follows:
\[
\mathbf{x}_i^{(t+1)} = \mathbf{x}_i^{(t)} + \mathbf{v}_i^{(t+1)}
\]

\subsection*{Objective Function}
After each iteration the particle's fitness value is calculated using the objective function. 

\subsection*{Best Positions}
\begin{itemize}
    \item The \textbf{personal best position}, $\mathbf{p}_i$, is updated whenever a particle achieves a better value for the objective function:
    \[
    \mathbf{p}_i = 
    \begin{cases} 
    \mathbf{x}_i^{(t+1)} & \text{if } f(\mathbf{x}_i^{(t+1)}) < f(\mathbf{p}_i) \\
    \mathbf{p}_i & \text{otherwise.}
    \end{cases}
    \]
    \item The \textbf{global best position}, $\mathbf{g}$, is updated to be the best position among all particles:
    \[
    \mathbf{g} = \arg\min_{i \in \{1, \dots, N\}} f(\mathbf{p}_i).
    \]
\end{itemize}

\subsection*{Stopping Criteria}
The above steps summarize the entire algorithm, however, when do we actually stop the iterations? In the assignment we are told that these are the following termination conditions: 
\begin{itemize}
    \item The maximum number of iterations, $T_{\text{max}}$, is reached.
    \item The fitness value of the global best position falls below a predefined threshold, $\epsilon$ which is double precision.
\end{itemize}
Notice that unlike the assignment I am using vector notation, thus eliminating the j components of the particle attributes. This was done simply to make the equations more compact, however, the code itself will use the scalar components of the vectors. 
	
  \begin{table}[H]
		\caption{\codebox{NUM\_VARIABLES = 10} (or dimension $d=10$) in \textbf{all} functions}
		\label{table:1}
		\centering
		\begin{tabular}{l c c c c c c c}
			\hline
			Function &  \multicolumn{2}{c}{Bound} & Particles & Iterations &  Optimal Fitness & CPU time (Sec) \\
			& Lower& Upper&&&\\
			\hline
			Griewank  		&  -600   & 600 	& 5000& 5000&  0.0196& 0.0387 &\\
			Levy 	  		&  -10    & 10 		&200 &1000 & 2.8179 &0.013285 &\\
			Rastrigin 		&  -5.12  & 5.12 	&2000 &1000 & 21.8890 & 0.011233&\\
			Rosenbrock		&  -5     & 10 		&50,000 &1000 & 0.2864 &2.1153 &\\
			Schwefel 	 	&  -500   & 500 	&50,000 &10,000 &1191.9687 &0.1291 &\\
			Dixon-Price 	&   -10	  & 10 		&200 &1000 &0.6667 &0.0080 &\\
			Michalewicz 	&   0 	  & $\pi$ 	&200000 &10,000 &-4.3275 &0.3551 &\\
			Styblinski-Tang & -5 	  & 5  		&20000 &100 &-320.9715 &0.0205 &\\
			\hline
		\end{tabular}
	\end{table}
	
	
	\begin{table}[H]
		\caption{\codebox{NUM\_VARIABLES = 50} (or dimension $d=50$) in \textbf{all} functions}
		\label{table:1}
		\centering
		\begin{tabular}{l c c c c c c c}
			\hline
			Function &  \multicolumn{2}{c}{Bound} & Particles & Iterations &  Optimal Fitness & CPU time (Sec) \\
			& Lower& Upper&&&\\
			\hline
			Griewank  & -600 & 600 & 50000 & 5000 & 0.000000 & 0.9745  \\
			Levy      & -10  & 10  & 10000 & 1000 & 13.7100 & 0.3406\\
			Rastrigin & -5.12 & 5.12 & 500000 & 10000 & 153.2942 & 3.5899 \\
			Rosenbrock& -5   & 10  & 900000 & 10000 & 25052.9603 & 2.8546 \\
			Schwefel  & -500 & 500 & 400000 & 10000 & 8944.0695 & 3.1870\\
			Dixon-Price & -10 & 10  & 20000 & 1000 & 321.0000 & 0.2272 \\
			Michalewicz & 0  & $\pi$ & 250000 & 10000 & -21.6375 & 3.4989  \\
			Styblinski-Tang & -5 & 5 & 90000 & 10000 & -1717.9840 &  2.2758 \\
			\hline
		\end{tabular}
	\end{table}
	
	\begin{table}[H]
		\caption{\codebox{NUM\_VARIABLES = 100} (or dimension $d=100$) in \textbf{all} functions}
		\label{table:3}
		\centering
		\begin{tabular}{l c c c c c c c}
			\hline
			Function &  \multicolumn{2}{c}{Bound} & Particles & Iterations &  Optimal Fitness & CPU time (Sec) \\
			& Lower& Upper&&&\\
			\hline
			Griewank  & -600 & 600 & 2000 & 1000 & 0.0034 & 0.6769 \\
			Levy      & -10  & 10  & 20000 & 10000 & 43.0579 & 6.4120 \\
			Rastrigin & -5.12 & 5.12 & 500000 & 10000 & 331.5333 & 6.9625 \\    
			Rosenbrock& -5   & 10  & 500000 & 10000 & 99.4779 & 4.4570 \\
			Schwefel  & -500 & 500 & 700000 & 10000 & 16241.0766 & 7.2933 \\
			Dixon-Price & -10 & 10  & 20000 & 2000 & 85.8870 & 1.3726 \\
			Michalewicz & 0  & $\pi$ & 300000 & 10000 & -48.6944 & 8.3429 \\
			Styblinski-Tang & -5 & 5 & 200000 & 10000 & -3450.1048 & 6.0242 \\
			\hline
		\end{tabular}
	\end{table}
	 
	\newpage
    \section{Running the Program}
    Simply run the Makefile by running "make" then you can run the executable pso by running: 
    \begin{lstlisting}
    ./pso <Function name> <Dimensions> <Lower Bound> <Upper Bound> <Number of Particles> <Max iterations> 
    \end{lstlisting}
	\section{Appendix}
    \subsection{PSO.c}
	
	\begin{lstlisting}[basicstyle=\small]
	// CODE: include library(s)
#include "utility.h"
#include "OF_lib.h"
#include <stdio.h> 
#include <stdlib.h> 
#include <math.h> 

// Helper function to generate random numbers in a range
double random_double(double min, double max) {
    return min + (max - min) * ((double)rand() / RAND_MAX);
}

// CODE: implement other functions here if necessary

// Function which will initialize a particle and all its attributes as seen in the psuedo code 
void initializer(Particle * particles, int NUM_PARTICLES, int NUM_VARIABLES, Bound * bounds, ObjectiveFunction objectiveFunction) {
    for (int i = 0; i < NUM_PARTICLES; i++) {
        // declare a new particle 
        Particle particle; 
        
        // allocate  memory for a particles attributes defined in the Particle struct
        particle.position = malloc(NUM_VARIABLES * sizeof(double)); 
        particle.velocity = malloc(NUM_VARIABLES * sizeof(double)); 
        particle.bestPosition = malloc(NUM_VARIABLES * sizeof(double)); 

        // initialize values for the particles above attributes 
        for (int j = 0; j < NUM_VARIABLES; j++) {
            particle.position[j] = random_double(bounds[j].lowerBound, bounds[j].upperBound); 
            particle.velocity[j] = 0.0; 
            // the bestPosition will be initialized in PSO 
        }

        // retrieve particles current position and initialize the best value as the current value
        particle.value = objectiveFunction(NUM_VARIABLES, particle.position);
        particle.bestValue = particle.value; 

        particles[i] = particle; // this particle has now been added to the list of particles
    }
}

// Function which will update a particles attributes throughout the algorithm 
void attributeUpdates(Particle * particles, int NUM_PARTICLES, int NUM_VARIABLES, double * bestPosition, double w, double c1, double c2, ObjectiveFunction objectiveFunction, Bound * bounds) {
    // declare and initialize the global best value 
    double globalBest = objectiveFunction(NUM_VARIABLES, bestPosition); 

    // update particle attributes 
    for (int i = 0; i < NUM_VARIABLES; i++) {
        for (int j = 0; j < NUM_VARIABLES; j++) {
            // velocity update 
            particles[i].velocity[j] = w * particles[i].velocity[j] + c1 * random_double(0, 1) * (particles[i].bestPosition[j] - particles[i].position[j]) + c2 * random_double(0, 1) * (bestPosition[j] - particles[i].position[j]); 
            
            // position update
            particles[i].position[j] += particles[i].velocity[j]; 

            // clamp the particles position within the bounds as defined in the PseudoCode 
            if (particles[i].position[j] < bounds[j].lowerBound) { 
                particles[i].position[j] = bounds[j].lowerBound;
            }
            else if (particles[i].position[j] > bounds[j].upperBound) {
                particles[i].position[j] = bounds[j].upperBound;
            }
        }
        
        // particle value update 
        particles[i].value = objectiveFunction(NUM_VARIABLES, particles[i].position); 

        // best value/position update if needed
        if (particles[i].value < particles[i].bestValue) {
            particles[i].bestValue = particles[i].value; 
            for (int k = 0; k < NUM_VARIABLES; k++) {
                particles[i].bestPosition[k] = particles[i].position[k]; 
            }
        }

        // global best value/position update if needed 
        if (particles[i].value < globalBest) {
            globalBest = particles[i].value; 
            for (int k = 0; k < NUM_VARIABLES; k++) {
                bestPosition[k] = particles[i].position[k]; 
            }
        }

    }

}

// pso implementation 
double pso(ObjectiveFunction objectiveFunction, int NUM_VARIABLES, Bound *bounds, int NUM_PARTICLES, int MAX_ITERATIONS, double *bestPosition) {
    // declare and initialize threshold to double precision
    double threshold = 1.0e-15; 

    // allocate memmory for all particles and initialize them with their attributes
    Particle * particles = malloc(NUM_PARTICLES * sizeof(Particle)); 
    initializer(particles, NUM_PARTICLES, NUM_VARIABLES, bounds, objectiveFunction); 

    // initialize the bestPosition to the first particles position (this is just a placeholder essentially) 
    for (int i = 0; i < NUM_VARIABLES ; i++) {
       bestPosition[i] = particles[0].position[i]; 
    }

    // update attributes of the particles 
    for (int i = 0; i < MAX_ITERATIONS; i++) {
        attributeUpdates(particles, NUM_PARTICLES,  NUM_VARIABLES, bestPosition, 0.7, 1.5, 1.5, objectiveFunction, bounds); 
    }

    // free the allocated memory 
    for (int i = 0; i < NUM_PARTICLES; i++) {
        free(particles[i].position); 
        free(particles[i].velocity); 
        free(particles[i].bestPosition); 
    }
    free(particles); 

    return objectiveFunction(NUM_VARIABLES, bestPosition); 
}
	\end{lstlisting}
	\subsection{utility.h}
    \begin{lstlisting}
#ifndef UTILITY_H
#define UTILITY_H

// Function pointer type for objective functions
typedef double (*ObjectiveFunction)(int, double *);

typedef struct Bound{
    double lowerBound;
    double upperBound;
}Bound;


// Function prototypes
double random_double(double min, double max);
double pso(ObjectiveFunction objective_function, int NUM_VARIABLES, Bound *bounds, int NUM_PARTICLES, int MAX_ITERATIONS, double best_position[]);

// CODE: declare other functions and structures if necessary
typedef struct Particle {
    double * position; 
    double * velocity; 
    double * bestPosition; 
    double bestValue;
    double value; 
} Particle; 

void initializer(Particle * particles, int NUM_PARTICLES, int NUM_VARIABLES, Bound * bounds, ObjectiveFunction objectiveFunction); 
void attributeUpdates(Particle * particles, int NUM_PARTICLES, int NUM_VARIABLES, double * bestPosition, double w, double c1, double c2, ObjectiveFunction objectiveFunction, Bound * bounds); 

#endif // UTILITY_H
    \end{lstlisting}
	
	
\end{document}
